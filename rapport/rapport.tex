\documentclass[a4paper,9pt]{article}

\usepackage[utf8]{inputenc}
\usepackage{graphicx}
\usepackage[francais]{babel}
%\usepackage{fullpage}
\usepackage{color}
\usepackage{array}

\begin{document}


\begin{titlepage}
\centering
\huge
\bfseries
Visualisation Scientifique\\[1\baselineskip]
\vspace{0.5cm}
\normalfont
\large
Interpolation et Visualisation de données avec Paraview\\[1\baselineskip]
	
\vspace{2cm}

\begin{figure}[!h]
\centering
\includegraphics[width=0.9\textwidth]{couverture.png}
\end{figure}

\vspace{1.5cm}

\centering
\bfseries
\normalfont

\begin{tabular}{r c l}
Benjamin Aupetit & & Nicolas Cousin
\end{tabular}	

\vspace{0.5cm}

Ensimag\\
\textbf{\today}
\end{titlepage}

\clearpage
\newpage

\tableofcontents

\clearpage
\newpage

% INTRODUCTION

\section{Introduction}
\label{sec:introduction}
Une des étapes cruciales pour l'observation d'un phénomène a toujours été de prendre des séries de mesures des données physiques associées à ce phénomène. Malheureusement, il n'est pas toujours possible de mesurer une quantité en n'importe quel point de l'espace, d'abord parce que ça coûte cher et d'autre part parce que la position où l'on souhaite faire la mesure n'est pas toujours accessible.\\ Par conséquent, il existe des méthodes d'interpolation permettant d'évaluer approximativement la valeur d'une donnée physique (température, densité, ...) en un point de l'espace en fonction des valeurs que l'on a mesuré sur des positions voisines. Ce document présente deux méthodes d'interpolation que nous avons implémenté, et nous proposons ensuite de visualiser leurs résultats sur paraview qui est un logiciel permettant de visualiser les données d'une grille régulière.

\section{Première Méthode : Shepard}
\label{sec:shepard}

\subsection{Description}
\label{subsec:shepard_description}

\subsection{Complexité}
\label{subsec:shepard_complexite}

\section{Deuxième Méthode : Multiquadriques de Hardy}
\label{sec:hardy}

\subsection{Description}
\label{subsec:hardy_description}

\subsection{Complexité}
\label{subsec:hardy_complexite}

\section{Implémentation}
\label{sec:implementation}

\subsection{Types utilisés}
\label{subsec:types}

\subsection{Implémentation de la méthode de Shepard}
\label{subsec:shepard_implementation}

\subsection{Implémentation de la méthode des multiquadriques de Hardy}
\label{subsec:hardy_implementation}

\section{Tests et Résultats}
\label{sec:tests_resultats}

\subsection{Visualisation de données avec Paraview}
\label{subsec:paraview}

\subsection{Présentation des méthodes de tests}
\label{subsec:presentation_tests}

\subsection{Validation de l'implémentation}
\label{subsec:validation_implementation}

\subsection{Tests Complexes}
\label{subsec:tests_complexes}

\subsection{Comparaison des deux méthodes]}
\label{subsec:comparaison_methodes}

\subsubsection{Temps de calcul}
\label{subsec:temps_calcul}

\subsubsection{Précision de l'interpolation}
\label{subsec:precision_interpolation}

\section{Conclusion}
\label{sec:conclusion}

\end{document}
